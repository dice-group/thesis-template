%%%%%%%%%
% General setup
%%%%%%%%%

% increase size of the page list in the glossary tables
\setlength{\glspagelistwidth}{0.2\linewidth}

% Should a glossary for RDF namespaces be created?
% This is suggested for theses that make use of IRIs.
\newcounter{useRDFNSGlossary}
\setcounter{useRDFNSGlossary}{1}

% Should a glossary for (formal) symbols be created?
% This is suggested for long, complex theses (like a PhD thesis).
% It might be too much for a master thesis
\newcounter{useSymbolsGlossary}
\setcounter{useSymbolsGlossary}{1}

%%%%%%%%%
% Acronyms
%%%%%%%%%

% Add more acronyms if needed

\newabbreviation{dns}{DNS}{Domain Name Service}
\newabbreviation{html}{HTML}{Hyper Text Markup Language}
\newabbreviation{http}{HTTP}{Hypertext Transfer Protocol}
\newabbreviation{https}{HTTPS}{Hyper Text Transfer Protocol Secure}
\newabbreviation{ip}{IP}{Internet Protocol}
\newabbreviation{iri}{IRI}{Internationalized Resource Identifier}
\newabbreviation{json-ld}{JSON-LD}{JavaScript Object Notation for Linked Data}
\newabbreviation{npmi}{NPMI}{Normalized Pointwise Mutual Information}
\newabbreviation{owl}{OWL}{Web Ontology Language}
\newabbreviation{pmi}{PMI}{Pointwise Mutual Information}
\newabbreviation{rdf}{RDF}{Resource Description Framework}
\newabbreviation{rdfa}{RDFa}{Resource Description Framework in Attributes}
\newabbreviation{rdfs}{RDFS}{Resource Description Framework Schema}
\newabbreviation{sparql}{SPARQL}{SPARQL Protocol And RDF Query Language}
\newabbreviation{url}{URL}{Uniform Resource Locator}
\newabbreviation{uri}{URI}{Uniform Resource Identifier}
\newabbreviation{urn}{URN}{Uniform Resource Name}
\newabbreviation{w3c}{W3C}{World Wide Web Consortium}
\newabbreviation{xml}{XML}{Extensible Markup Language}
\newabbreviation{xsd}{XSD}{XML Schema Datatypes}

%%%%%%%%%
% RDF namespaces
%%%%%%%%%

% Check whether an RDF namespace glossary is needed
\ifthenelse{\value{useRDFNSGlossary}>0}{

% Define RDF namespace glossary
\newglossary*{namespaces}{RDF Namespaces}
% Define a command for easy definition of RDF namespaces
\newcommand*{\newRDFNS}[2]{
% Define glossary entry by adding "ns" at the end
% (important to avoid collisions with other glossaries)
\newglossaryentry{{#1}ns} {
  type={namespaces},
  name={\texttt{#1}},
  description={\url{#2}},
  sort={#1}
}
}
% Define the \iri{} command for an easy usage of namespaces
\newcommand*{\iri}[2]{
\ifx&#2& % #2 is empty; simply print the prefix
\texttt{\gls{{#1}ns}}
\else    % else, print the prefixed IRI
\texttt{\gls{{#1}ns}:#2}
\fi
}

% Define additional RDF namespaces below this line if needed

\newRDFNS{owl}{http://www.w3.org/2002/07/owl\#}
\newRDFNS{rdf}{http://www.w3.org/1999/02/22-rdf-syntax-ns\#}
\newRDFNS{rdfs}{http://www.w3.org/2000/01/rdf-schema\#}
\newRDFNS{xsd}{http://www.w3.org/2001/XMLSchema\#}

}{
% no RDF Namespaces, so nothing to do
}


%%%%%%%%%
% Math symbols
% (follows the idea at https://tex.stackexchange.com/questions/137418/how-to-create-a-list-of-symbols-where-symbols-can-be-used-in-math-mode
%%%%%%%%%

% Check whether an RDF namespace glossary is needed
\ifthenelse{\value{useRDFNSGlossary}>0}{

% From here on, all glossary entities are handled as symbols.
% This makes the definition of them easier.
\renewcommand{\glsdefaulttype}{symbols}

% For using symbols within equations 
% (symbols are defined further below)
\newcommand*{\s}[1]{\ensuremath{\gls*{#1}}}

% Sorting (just a suggestion, any other order is fine as well)
% operators: 0
% others:    1
% greek:     3
% latin:     4
% within the name
% lowercase: 1
% uppercase: 2

% Example symbols (feel free to remove them)
\newglossaryentry{avoidZeroLog} {
  name={\ensuremath{\epsilon}},
  description={A small constant that is added to avoid the calculation of the logarithm of 0},
  sort={3epsilon1avoidzerolog}
}
\newglossaryentry{randVar1} {
  name={\ensuremath{\mathcal{X}}},
  description={A random variable},
  sort={4x2rand}
}
\newglossaryentry{randValue1} {
  name={\ensuremath{x}},
  description={A random value of \gls{randVar1}},
  sort={4x1rand}
}
\newglossaryentry{randVar2} {
  name={\ensuremath{\mathcal{Y}}},
  description={A random variable},
  sort={4y2rand}
}
\newglossaryentry{randValue2} {
  name={\ensuremath{y}},
  description={A random value of \gls{randVar2}},
  sort={4y1rand}
}
\newglossaryentry{pmiMath} {
  name={\ensuremath{\text{PMI}}},
  description={The pointwise mutual information measure as defined in Equation~\ref{eq:pmi}},
  sort={4p2PMI}
}
\newglossaryentry{pmiMathE} {
  name={\ensuremath{\text{PMI}_\epsilon}},
  description={The pointwise mutual information measure for probabilities based on counts as defined in Equation~\ref{eq:pmi-e}},
  sort={4p2PMIe}
}
\newglossaryentry{npmiMath} {
  name={\ensuremath{\text{NPMI}}},
  description={The normalized pointwise mutual information measure as defined in Equation~\ref{eq:npmi}},
  sort={4n2NPMI}
}
\newglossaryentry{npmiMathE} {
  name={\ensuremath{\text{NPMI}_\epsilon}},
  description={The normalized pointwise mutual information measure for probabilities based on counts as defined in Equation~\ref{eq:npmi-e}},
  sort={4n2NPMIe}
}
\newglossaryentry{probability} {
  name={\ensuremath{\mathbb{P}}},
  description={A probability},
  sort={4p2prob}
}

% Add additional symbol definition here


}{
% No glossary for formal symbols, so nothing to do
}



% Necessary command to prepare glossary definition
\makeglossaries