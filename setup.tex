% **************************************************
% Files' Character Encoding
% **************************************************
%\PassOptionsToPackage{utf8}{inputenc}
\usepackage[utf8]{inputenc}
\usepackage[T2A]{fontenc}
\usepackage[main=english]{babel} % babel system, adjust the language of the content


% **************************************************
% Information and Commands for Reuse
% **************************************************
\newcommand{\thesisTitle}{Amazing Thesis Title} % The title of the thesis
\newcommand{\thesisName}{Max Power} % The name of the author
\newcommand{\thesisType}{Bachelor Thesis}% "Bachelor Thesis", "Master Thesis", or "Doctoral Dissertation"
\newcommand{\thesisTypeShort}{Thesis}% "Thesis" for a Bachelor or Master thesis, or "Dissertation" for a PhD thesis.
\newcommand{\thesisDegree}{Bachelor of Science} % "Bachelor of Science", "Master of Science", or the PhD degree, e.g., "Dr.rer.nat."
\newcommand{\thesisDate}{01.01.1970} % date of submission
\newcommand{\thesisVersion}{Template (0.0.0)} % Version (typically, the submission is version 1.0.0; but it is up to you

% Add your first reviewer (in the DICE group, this is typically Prof. Ngonga
\newcommand{\thesisFirstReviewer}{Prof. Dr. Axel-Cyrille Ngonga Ngomo}
\newcommand{\thesisFirstReviewerUniversity}{\protect{Paderborn University}}
\newcommand{\thesisFirstReviewerDepartment}{Department of Computer Science}

% Add your second reviewer
\newcommand{\thesisSecondReviewer}{Prof. Dr. Second Examiner}
\newcommand{\thesisSecondReviewerUniversity}{Clean Thesis Style University}
\newcommand{\thesisSecondReviewerDepartment}{Department of Computer Science}

% Who supervised your work? Typically, it is Prof. Ngonga and somebody from his team
\newcommand{\thesisFirstSupervisor}{Prof. Dr. Axel-Cyrille Ngonga Ngomo}
\newcommand{\thesisSecondSupervisor}{Dr. Amazing Researcher} % Please add here the name of your second supervisor

% The last values are standard values and should hold for theses written with the DICE group
\newcommand{\thesisUniversity}{\protect{Paderborn University}}
\newcommand{\thesisUniversityDepartment}{Faculty of Computer Science, Electrical Engineering and Mathematics}
\newcommand{\thesisUniversityInstitute}{Department of Computer Science}
\newcommand{\thesisUniversityGroup}{Data Science Group (DICE)}
\newcommand{\thesisUniversityCity}{Paderborn}
\newcommand{\thesisUniversityStreetAddress}{Warburger Str. 100}
\newcommand{\thesisUniversityPostalCode}{33098}


% **************************************************
% Debug LaTeX Information
% **************************************************
%\listfiles


% **************************************************
% Load and Configure Packages
% **************************************************
%%% Start with packages that we want to configure and that would be loaded by the cleanthesis package without any configuration 
% Compact list / enumerate format
\usepackage[inline]{enumitem}
\setlist{noitemsep,topsep=0pt,parsep=0pt,partopsep=0pt}
\setenumerate{noitemsep,topsep=0pt,parsep=0pt,partopsep=0pt}

%%% Load cleanthesis package
\PassOptionsToPackage{% setup clean thesis style
    figuresep=colon,%
    hangfigurecaption=false,%
    hangsection=false,%
    hangsubsection=false,%
    sansserif=false,%
    configurelistings=true,%
    colorize=full,%
    colortheme=bluemagenta,%
    configurebiblatex=false
}{cleanthesis}
\usepackage{cleanthesis}

\hypersetup{% setup the hyperref-package options
    pdftitle={\thesisTitle},    %   - title (PDF meta)
    pdfsubject={\thesisType},   %   - subject (PDF meta)
    pdfauthor={\thesisName},    %   - author (PDF meta)
    plainpages=false,           %   -
    colorlinks=false,           %   - colorize links?
    pdfborder={0 0 0},          %   -
    breaklinks=true,            %   - allow line break inside links
    bookmarksnumbered=true,     %
    bookmarksopen=true          %
}

%%% Load other packages
\usepackage{amsmath, amssymb, amsfonts}% mathematical symbols and the like
\usepackage{mathrsfs} %mathscr font
\usepackage{amsthm}% definitions, theorems, etc.
\usepackage[colorinlistoftodos,disable]{todonotes}% marking open todos in text/on margins
\usepackage{subfig}% multi-part figures with separate captions per part
\usepackage{url}% render URLs correctly and make them clickable through the hyperref package
\usepackage{longtable}% tables that span multiple pages
\usepackage{booktabs}% tables that actually look good
%\usepackage[nolist]{acronym}% consistently use acronyms
\RequirePackage{comment}
\usepackage[nottoc,notlot,notlof]{tocbibind} %Not recommended for KOMA script % Adds lists of figures, tables, ..., algorithms to the toc. Should be loaded before algorithm2e
\usepackage[linesnumbered,ruled,vlined,algochapter]{algorithm2e} % dotocloa would add an entry to the ToC
\usepackage{multirow}
\usepackage{hyperref}
%\usepackage[order=letter,acronym,symbols,nomain]{glossaries}
\usepackage[automake,order=letter,acronym,symbols,postdot,nomain,stylemods]{glossaries-extra} % already contains the glossaries package
\usepackage{glossary-longbooktabs}
\usepackage{listings}
\usepackage{rotating}
\usepackage{ifthen} % for if else constructs

\usepackage{footnote}
\usepackage{footmisc}

\usepackage[sort,square,numbers]{natbib}% more bibliography options

\usepackage[dvipsnames]{xcolor}
\usepackage{graphicx}

\usepackage{tikz}
\usetikzlibrary{bayesnet}
\usetikzlibrary{arrows}
\usetikzlibrary{graphs}
\usetikzlibrary{positioning,calc}
\usetikzlibrary{decorations.pathreplacing}

\usepackage{pgfplots}
\usetikzlibrary{arrows.meta}
\usetikzlibrary{decorations.markings}
\usepgfplotslibrary{statistics} % boxplots


\usepackage{lscape} % landscape pages
\usepackage{afterpage} % handling floating-like behavior of landscape pages

\usepackage{float}
\usepackage{placeins} % Placeins.sty keeps floats ‘in their place’, preventing them from floating past a “\FloatBarrier” command into another section. To use it, declare “\usepackage{placeins}” and insert “\FloatBarrier” at places that floats should not move past, perhaps at every “\section”

% Chapter counters should be resetted if we change the part counter
%\usepackage{chngcntr}
%\counterwithin*{chapter}{part}

%%%%%%%%%
% Some commands for setting up theorem environments as provided by package 
% amsthm --- language sensitive
%%%%%%%%%
\theoremstyle{plain}
\newtheorem{definition}{Definition}[chapter]
\newtheorem{lemma}[definition]{Lemma}
%\ifgerman
%	\newtheorem{theorem}[definition]{Satz}
%	\newtheorem{corollary}[definition]{Korollar}
%	\newtheorem{example}[definition]{Beispiel}
%\else
	\newtheorem{theorem}[definition]{Theorem}
	\newtheorem{corollary}[definition]{Corollary}
	\newtheorem{example}[definition]{Example}
%\fi

%%%%%%%%
% Further template commands
%%%%%%%%
% Command to handle header and footer of empty pages
\newcommand*{\addpageifneeded}{\Ifthispageodd{\newpage $\quad$ \cleardoubleoddpage}{\clearpage}}