\chapter{Evaluation}
\label{ch:evaluation}

The evaluation chapter is an important chapters and its classic structure is described in the following. The chapter starts with a description of goals that the whole evaluation has. This can be combined with research questions that might have been raised in the introdcution chapter. Or it is simply the statement that the effectiveness and efficiency of the proposed approach should be measured. If the approach is very general but the evaluation focusses on a certain use case, this should be described here as well (e.g., a generic recommendation approach is proposed but the evaluation focusses on the recommendation of movies).

\section{Setup}

The first part of the evaluation chapter describes the experiments which will be carried out. The setup includes all details that would be necessary to repeat the evaluation. In addition, it should give additional details, e.g., about the data that is used (e.g., size of the dataset).
This typically includes the following:
\begin{itemize}
\item A description of the experiment that will be carried out. If there are several experiments, all of them should be described.
\item A description of the dataset that is used.
\item A description of the evaluation measures (if they haven't been defined in the Background chapter).
\end{itemize}

\section{Results}

This section comprises the results of all experiments. If the author decides to separate the discussion from the results, it is important that this section only describes the results without interpreting them.

\section{Discussion}

This section contains the discussion of the evaluation results. This discussion can be either in this chapter, within an own chapter, or part of the summary chapter. Wherever the discussion will be located, it is important that the discussion goes beyond a simple description of the measured results. The author should try to interpret them. At the same time, it is important to avoid statements which cannot be proven based on the evaluation results.